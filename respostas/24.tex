\subsection{c} Escreva um pseudo-código de um algoritmo de complexidade $O(V +E)$ que recebe um grafo orientado acíclico $G$ representado por listas de adjacências e um vetor \textbf{cor} de cores e devolve um vetor $val[~]$ indexado por $V$ tal que $val[u]$ é o número de caminhos válidos que começam em $u$ para cada $u \in V[G]$.

\itemdsep[0.25]


\newcommand{\Azul}{\const{azul}\xspace}
\newcommand{\Vermelho}{\const{vermelho}\xspace}

\begin{codebox}
\Procname{$\proc{Caminhos-Válidos}(G, \cor)$}
\li \Sejam $\azul[~]$ e $\verm[~]$ vetores indexados pelos vértices em $V[G]$.
\li \Para $v \in V[G]$ \Faca        \label{linha:2c:init}
    \Do
\li     $\azul[v] \Recebe \Nulo$
\li     $\verm[v] \Recebe \Nulo$    \label{linha:2c:init:end}
    \End
\li
\li \Seja $val[~]$ um vetor também indexado por $V[G]$
\li \Para $v \in V[G]$ \Faca        \label{linha:2c:preenche}
    \Do
\li     $\proc{Caminhos-Azuis-Vermelhos}(v, \cor, \azul, \verm)$    \label{linha:2c:preenche:chamada}
\li     $val[v] = \azul[v] + \verm[v]$  \label{linha:2c:preenche:end}
    \End
\li \Retorna $val$
\end{codebox}

\begin{codebox}
\Procname{$\proc{Caminhos-Azuis-Vermelhos}(v, \cor, \azul, \verm)$}
\li \Se $\azul[v] \ne \Nulo$ e $\verm[v] \ne \Nulo$ \label{linha:2c:memo}
    \Do
\li     \Entao \Retorna
    \End
\li
\li $\azul[v] \Recebe 0$
\li $\verm[v] \Recebe 0$
\li \Para $u \in \Adj[v]$ \Faca \label{linha:2c:calc}
    \Do
\li     $\proc{Caminhos-Azuis-Vermelhos}(u, \cor, \azul, \verm)$ \label{linha:2c:calc:chamada}
\li     \Se $\cor[v,u] = \Azul$
        \Do
\li         \Entao $\azul[v] \Recebe \azul[v] + 1 + \azul[u] + \verm[u]$
\li         \Senao $\verm[v] \Recebe \verm[v] + 1 + \azul[u]$ \label{linha:2c:calc:end}
        \End
    \End
\end{codebox}
