\subsection{c} Escreva um pseudo-código de um algoritmo de complexidade $O(V +E)$ que recebe um grafo orientado acíclico $G$ representado por listas de adjacências e um vetor \textbf{cor} de cores e devolve um vetor $val[~]$ indexado por $V$ tal que $val[u]$ é o número de caminhos válidos que começam em $u$ para cada $u \in V[G]$.

\itemdsep[0.25]


\newcommand{\Azul}{\const{azul}\xspace}
\newcommand{\Vermelho}{\const{vermelho}\xspace}

\begin{codebox}
\Procname{$\proc{Caminhos-Válidos}(G, \cor)$}
\li \Sejam $\azul[~]$ e $\verm[~]$ vetores indexados pelos vértices em $V[G]$
\li \Para $u \in V[G]$ \Faca        \label{linha:2c:init}
    \Do
\li     $\azul[u] \Recebe \Nulo$    \label{linha:2c:init:end}
        \>\>\>\>\>\Comment usado para marcar vértices não visitados
    \End
\li
\li \Seja $val[~]$ um vetor também indexado por $V[G]$
\li \Para $u \in V[G]$ \Faca        \label{linha:2c:preenche}
    \Do
\li     \Se $\azul[u] = \Nulo$ \Entao
        \Do
\li         $\proc{Caminhos-Azuis-Vermelhos}(u, \cor, \azul, \verm)$    \label{linha:2c:preenche:chamada}
        \End
\li
\li     $val[u] = 1 + \azul[u] + \verm[u]$  \label{linha:2c:preenche:end}
\li     \Comment caminho trivial $(u)$, mais os válidos começando com azul ou vermelho
    \End
\li \Retorna $val$
\end{codebox}

A função $\proc{Caminhos-Azuis-Vermelhos}$ abaixo calcula os valores de $\azul[u]$ e $\verm[u]$ pelas relações encontradas no \textref{sec:2b}{item anterior}. Para isso, pode ser necessário ir preenchendo os vetores onde o valor ainda não foi calculado.

\begin{codebox}
\Procname{$\proc{Caminhos-Azuis-Vermelhos}(u, \cor, \azul, \verm)$}
\li $azul \Recebe 0$
\li $verm \Recebe 0$
\li \Para $v \in \Adj[u]$ \Faca \label{linha:2c:calc}
    \Do
\li     \Se $\azul[v] = \Nulo$ \Entao
        \Do
\li         $\proc{Caminhos-Azuis-Vermelhos}(v, \cor, \azul, \verm)$    \label{linha:2c:calc:chamada}
        \End
\li
\li     \Se $\cor[u,v] = \Azul$
        \>\>\>\>\>\>\>\Comment do item 2b
        \Do
\li         \Entao $azul \Recebe azul + 1 + \azul[v] + \verm[v]$
\li         \Senao $verm \Recebe verm + 1 + \azul[v]$ \label{linha:2c:calc:end}
        \End
    \End
\li
\li \Comment salva os valores
\li $\azul[v] \Recebe azul$
\li $\verm[v] \Recebe verm$
\end{codebox}
