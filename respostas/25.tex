\subsection{d} Justifique a complexidade do seu algoritmo do item (c).

\itemdsep[0.25]

O procedimento $\proc{Caminhos-Azuis-Vermelhos}$ é chamado no máximo uma vez por vértice $u \in V[G]$. Em cada chamada, o laço das linhas \ref{linha:2c:calc} a \ref{linha:2c:calc:end} executa $\abs{\Adj[u]}$ vezes. Então, os vetores $\azul[~]$ e $\verm[~]$ são preenchidos com tempo total
\[
    \sum_{u \in V} O\left(\abs{\Adj[u]}\right) = O\left(\sum_{u \in V} \abs{\Adj[u]}\right) = O(E)
\]

Além disso, $\proc{Caminhos-Válidos}$ tem os laços das linhas \ref{linha:2c:init} e \ref{linha:2c:preenche}, que executam uma vez por nó, ou seja, em tempo $O(V)$. A complexidade total deve ser, então, $O(V) + O(E) = O(V + E)$.
