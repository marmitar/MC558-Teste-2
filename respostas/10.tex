Modifique o pseudo-código do algoritmo de busca em profundidade apresentado em aula ou do CLRS (supondo que o grafo de entrada $G$ é orientado) para imprimir cada aresta $(u, v)$ juntamente com seu tipo (aresta da árvore, de avanço, de retorno ou de cruzamento). A complexidade do DFS modificado ainda dever ser $O(V + E)$.

\itemdsep

\newcommand{\Branco}{\const{branco}\xspace}
\newcommand{\Cinza}{\const{cinza}\xspace}
\newcommand{\Preto}{\const{preto}\xspace}

\begin{codebox}
\Procname{$\proc{DFS}(G)$}
\li \Para \Cada $u \in V[G]$ \Faca
    \Do
\li     $cor[u] \Recebe \Branco$
\li     $\pi[u] \Recebe \Nulo$
    \End
\li $tempo \Recebe 0$
\li \Para \Cada $u \in V[G]$ \Faca
    \Do
\li     \Se $cor[u] = \Branco$
        \Do
\li         \Entao $\proc{DFS-Visit}(u)$
        \End
    \End
\end{codebox}

\begin{codebox}
\Procname{$\proc{DFS-Visit}(u)$}
\li $cor[u] \Recebe \Cinza$
\li $tempo \Recebe tempo + 1$
\li $d[u] \Recebe tempo$
\li \Para \Cada $v \in \Adj[u]$
    \Do
\li     \Se $cor[v] = \Branco$ \Entao
\li     \Comment próximo vértice a ser visitado, então $uv$ está na floresta BP
        \Do
\li         $\proc{Imprime}(u, v, \text{``aresta da árvore''})$
\li
\li         $\pi[v] \Recebe u$
\li         $\proc{DFS-Visit}(v)$
        \End
\li     \Senao, \Se $cor[v] = \Cinza$ \Entao
\li     \Comment $v$ ainda não terminou, então ele é algum ancestral de $u$
        \Do
\li         $\proc{Imprime}(u, v, \text{``aresta de retorno''})$
        \End
\li     \Senao, \Se $d[u] < d[v]$ \Entao
\li     \Comment $v$ já foi visitado, mas antes de terminar $u$, então ainda é descendente de $u$
        \Do
\li         $\proc{Imprime}(u, v, \text{``aresta de avanço''})$
        \End
\li     \Senao
        \Do
\li         $\proc{Imprime}(u, v, \text{``aresta de cruzamento''})$
        \End
    \End
\li $cor[u] \Recebe \Preto$
\li $tempo \Recebe tempo + 1$
\li $f[u] \Recebe tempo$
\end{codebox}
