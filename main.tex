\documentclass[a4paper, 14pt]{extarticle}

%% Pacotes Gerais %%
\usepackage[utf8]{inputenc}
\usepackage[T1]{fontenc}
\usepackage[brazilian]{babel}


%%%%%%%%%%%%%%%%%%%%%%%%%%%%%
%% Formatação do Documento %%

\usepackage{geometry}
\geometry{
    margin = 1.5cm,
    noheadfoot = true
}


%%%%%%%%%%%%%%%%%
%% Referências %%
\usepackage{nameref, xcolor, url}

\usepackage{hyperref}
\hypersetup{
    pdftitle  = {MC558 - Teste 2},
    pdfauthor = {Tiago de Paula Alves},
    % bookmarks   = true,
    pdfpagemode = UseOutlines,
    %% Cores de Links %%
    colorlinks = true,
    linkcolor  = blue!30!black,
    urlcolor   = red!30!black,
    citecolor  = blue
}

% 'hyperref' com substituição
\newcommand{\textref}[3][\ref]{{%
    \def\swaptext##1{#3}%
    \hyperref[#2]{\swaptext{#1*{#2}}}%
}}

\newcommand{\equref}[2][\ref]{%
    \textref[#1]{#2}{(##1)}
}

\renewcommand{\url}[1]{
    \href{#1}{\texttt{#1}}
}


%%%%%%%%%%%%%%%%%%%%%%
%% Opções de Seções %%
\usepackage{titlesec, fancyhdr}

% Formatação de seção e subseção
\titleformat{\section}[runin]
    {\titlerule{}\vspace{1ex}\normalfont\large\bfseries}
    {}{.5em}{}[.]
\titleformat{\subsection}[runin]
    {\normalfont\normalsize\bfseries}
    {}{1em}{}[:]

% Separador das seções
\newcommand{\docline}[1][\pagebreak]{%
    ~ \\
    \noindent\rule{\textwidth}{0.4pt}%
    #1
}
% Separador de itens ou subseções
\newcommand{\itemdsep}{
    \noindent\hfil\rule{0.5\textwidth}{.2pt}\hfil
    \vskip1em
}

% Páginas sem numeração
\pagestyle{empty}


%%%%%%%%%%%%%%%%%%%%%
%% Opções do Título %%
\usepackage{titling}

% Título mais pra cima
\pretitle{%
    \vspace{-6em}%
    \begin{center}%
        \Large%
}
\posttitle{%
    \end{center}%
}
% Reduz separação do autor
\preauthor{%
    \vspace{-1.5em}
    \begin{center}%
        \begin{tabular}[t]{c}
}
\postauthor{%
        \end{tabular}%
    \end{center}%
    \vspace{-1.5em}
}
% Sem data
\predate{}\date{}\postdate{}

% Título e autor
\title{
    {\normalsize  MC558 2020s1} \\
    {\LARGE       Teste 2}
}
\author{
    {\normalsize  Tiago de Paula Alves} \\
    {\small       187679}
}


%%%%%%%%%%%%%%%%
%% Documentos %%

\input{teoremas}
\input{simbolos}
\usepackage{clrscode3e, xspace}


%% KEYWORDS %%

\newcommand{\Para}{\kw{para}\xspace}
\newcommand{\Ate}{\kw{até}\xspace}
\newcommand{\DAte}{\kw{descendo} \Ate\xspace}
% \By
\newcommand{\Enquanto}{\kw{enquanto}\xspace}
\newcommand{\Se}{\kw{se}\xspace}
\newcommand{\Retorna}{\kw{retorna}\xspace}
\newcommand{\VaPara}{\kw{vá para}\xspace}
\newcommand{\Erro}{\kw{erro}\xspace}
% \Spawn
% \Sync
% \Parfor


% cleveref depois dos outros pacotes
\usepackage[nameinlink,noabbrev,brazilian]{cleveref}

\begin{document}
    \maketitle
    \thispagestyle{empty}

    \section{1}
    \begingroup
        Modifique o pseudo-código do algoritmo de busca em profundidade apresentado em aula ou do CLRS (supondo que o grafo de entrada $G$ é orientado) para imprimir cada aresta $(u, v)$ juntamente com seu tipo (aresta da árvore, de avanço, de retorno ou de cruzamento). A complexidade do DFS modificado ainda dever ser $O(V + E)$.

\itemdsep

\newcommand{\Branco}{\const{branco}\xspace}
\newcommand{\Cinza}{\const{cinza}\xspace}
\newcommand{\Preto}{\const{preto}\xspace}

\begin{codebox}
\Procname{$\proc{DFS}(G)$}
\li \Para \Cada $u \in V[G]$ \Faca
    \Do
\li     $cor[u] \Recebe \Branco$
\li     $\pi[u] \Recebe \Nulo$
    \End
\li $tempo \Recebe 0$
\li \Para \Cada $u \in V[G]$ \Faca
    \Do
\li     \Se $cor[u] = \Branco$
        \Do
\li         \Entao $\proc{DFS-Visit}(u)$
        \End
    \End
\end{codebox}

\begin{codebox}
\Procname{$\proc{DFS-Visit}(u)$}
\li $cor[u] \Recebe \Cinza$
\li $tempo \Recebe tempo + 1$
\li $d[u] \Recebe tempo$
\li \Para \Cada $v \in \Adj[u]$
    \Do
\li     \Se $cor[v] = \Branco$ \Entao
\li     \Comment próximo vértice a ser visitado, então $uv$ está na floresta BP
        \Do
\li         $\proc{Imprime}(u, v, \text{``aresta da árvore''})$
\li
\li         $\pi[v] \Recebe u$
\li         $\proc{DFS-Visit}(v)$
        \End
\li     \Senao, \Se $cor[v] = \Cinza$ \Entao
\li     \Comment $v$ ainda não terminou, então ele é algum ancestral de $u$
        \Do
\li         $\proc{Imprime}(u, v, \text{``aresta de retorno''})$
        \End
\li     \Senao, \Se $d[u] < d[v]$ \Entao
\li     \Comment $v$ já foi visitado, mas antes de terminar $u$, então ainda é descendente de $u$
        \Do
\li         $\proc{Imprime}(u, v, \text{``aresta de avanço''})$
        \End
\li     \Senao
        \Do
\li         $\proc{Imprime}(u, v, \text{``aresta de cruzamento''})$
        \End
    \End
\li $cor[u] \Recebe \Preto$
\li $tempo \Recebe tempo + 1$
\li $f[u] \Recebe tempo$
\end{codebox}

    \endgroup
    \docline

    \section{2}
    \begingroup
        Seja $G$ um grafo orientado acíclico. Suponha que cada aresta $(u, v) \in E[G]$ tem uma cor $cor(u, v)$ que pode ser azul ou vermelha. Um caminho $P$ em $G$ é válido se não possui arestas consecutivas de cor vermelha. [...]

Nesta questão, você deve projetar um algoritmo linear que para cada vértice $u \in V[G]$, devolve o número de caminhos válidos que começam em u.

\newcommand{\azul}{\mathrm{azul}\xspace}
\newcommand{\verm}{\mathrm{verm}\xspace}

\begin{definition*}
    Defina $\azul[u]$ (respectivamente, $\verm[u]$) como o número de caminhos válidos com início em $u$ cuja primeira aresta tem cor azul (respectivamente, vermelha). Note que o caminho trivial válido $(u)$ não contribui para nenhum desses valores.
\end{definition*}

\itemdsep

\begin{figure}[H]
    \centering
    %% Creator: Inkscape 1.0.2 (e86c870879, 2021-01-15, custom), www.inkscape.org
%% PDF/EPS/PS + LaTeX output extension by Johan Engelen, 2010
%% Accompanies image file '21_grafo.pdf' (pdf, eps, ps)
%%
%% To include the image in your LaTeX document, write
%%   \input{<filename>.pdf_tex}
%%  instead of
%%   \includegraphics{<filename>.pdf}
%% To scale the image, write
%%   \def\svgwidth{<desired width>}
%%   \input{<filename>.pdf_tex}
%%  instead of
%%   \includegraphics[width=<desired width>]{<filename>.pdf}
%%
%% Images with a different path to the parent latex file can
%% be accessed with the `import' package (which may need to be
%% installed) using
%%   \usepackage{import}
%% in the preamble, and then including the image with
%%   \import{<path to file>}{<filename>.pdf_tex}
%% Alternatively, one can specify
%%   \graphicspath{{<path to file>/}}
%%
%% For more information, please see info/svg-inkscape on CTAN:
%%   http://tug.ctan.org/tex-archive/info/svg-inkscape
%%
\begingroup%
  \makeatletter%
  \providecommand\color[2][]{%
    \errmessage{(Inkscape) Color is used for the text in Inkscape, but the package 'color.sty' is not loaded}%
    \renewcommand\color[2][]{}%
  }%
  \providecommand\transparent[1]{%
    \errmessage{(Inkscape) Transparency is used (non-zero) for the text in Inkscape, but the package 'transparent.sty' is not loaded}%
    \renewcommand\transparent[1]{}%
  }%
  \providecommand\rotatebox[2]{#2}%
  \newcommand*\fsize{\dimexpr\f@size pt\relax}%
  \newcommand*\lineheight[1]{\fontsize{\fsize}{#1\fsize}\selectfont}%
  \ifx\svgwidth\undefined%
    \setlength{\unitlength}{460.22099304bp}%
    \ifx\svgscale\undefined%
      \relax%
    \else%
      \setlength{\unitlength}{\unitlength * \real{\svgscale}}%
    \fi%
  \else%
    \setlength{\unitlength}{\svgwidth}%
  \fi%
  \global\let\svgwidth\undefined%
  \global\let\svgscale\undefined%
  \makeatother%
  \begin{picture}(1,0.19870508)%
    \lineheight{1}%
    \setlength\tabcolsep{0pt}%
    \put(0,0){\includegraphics[width=\unitlength,page=1]{respostas/21_grafo.pdf}}%
    \put(0.98846278,0.06679875){\color[rgb]{0,0,0}\makebox(0,0)[lt]{\lineheight{1.25}\smash{\begin{tabular}[t]{l}$z$\end{tabular}}}}%
    \put(-0.0020552,0.06702944){\color[rgb]{0,0,0}\makebox(0,0)[lt]{\lineheight{1.25}\smash{\begin{tabular}[t]{l}$p$\end{tabular}}}}%
    \put(0.09591444,0.06701983){\color[rgb]{0,0,0}\makebox(0,0)[lt]{\lineheight{1.25}\smash{\begin{tabular}[t]{l}$q$\end{tabular}}}}%
    \put(0.19245584,0.06686864){\color[rgb]{0,0,0}\makebox(0,0)[lt]{\lineheight{1.25}\smash{\begin{tabular}[t]{l}$r$\end{tabular}}}}%
    \put(0.29154197,0.05085898){\color[rgb]{0,0,0}\makebox(0,0)[lt]{\lineheight{1.25}\smash{\begin{tabular}[t]{l}$s$\end{tabular}}}}%
    \put(0.38770429,0.06722658){\color[rgb]{0,0,0}\makebox(0,0)[lt]{\lineheight{1.25}\smash{\begin{tabular}[t]{l}$t$\end{tabular}}}}%
    \put(0.48495918,0.06684231){\color[rgb]{0,0,0}\makebox(0,0)[lt]{\lineheight{1.25}\smash{\begin{tabular}[t]{l}$u$\end{tabular}}}}%
    \put(0.58256316,0.06709827){\color[rgb]{0,0,0}\makebox(0,0)[lt]{\lineheight{1.25}\smash{\begin{tabular}[t]{l}$v$\end{tabular}}}}%
    \put(0.68017119,0.05058472){\color[rgb]{0,0,0}\makebox(0,0)[lt]{\lineheight{1.25}\smash{\begin{tabular}[t]{l}$w$\end{tabular}}}}%
    \put(0.77719442,0.05044547){\color[rgb]{0,0,0}\makebox(0,0)[lt]{\lineheight{1.25}\smash{\begin{tabular}[t]{l}$x$\end{tabular}}}}%
    \put(0.89065104,0.06694284){\color[rgb]{0,0,0}\makebox(0,0)[lt]{\lineheight{1.25}\smash{\begin{tabular}[t]{l}$y$\end{tabular}}}}%
  \end{picture}%
\endgroup%


    \caption{Para quem tem dificuldades de distinguir as cores, as arestas vermelhas são: $(p, q)$, $(q, r)$, $(p, s)$, $(s, x)$, $(t, x)$, $(u, v)$, $(u, w)$, $(w, x)$ e $(x, y)$. Caminhos de comprimento zero ou um são sempre válidos. Os caminhos $(q, s, t, x, z)$ e $(p, s, t, w, y)$ também são válidos. Já o caminho $(p, q, s, t, x, y, z)$ não é válido pois $(t, x)$ e $(x, y)$ são arestas consecutivas de cor vermelha neste caminho.}
\end{figure}

\subsection{a} Para cada vértice $i$ do grafo acima, indique os valores $\azul[i]$ e $\verm[i]$ (alguns valores estão preenchidos).

\itemdsep[0.25]

\begin{table}[H]
    \centering
    \begin{tabular}{|c|c|c|c|c|c|c|c|c|c|c|c|}
        \hline
        $i$ & $p$ & $q$ & $r$ & $s$ & $t$ & $u$ & $v$ & $w$ & $x$ & $y$ & $z$ \\
        \hline
        $\azul[i]$ & & & & & & 2 & 0 & 2 & 1 & 1 & 0 \\
        \hline
        $\verm[i]$ & & & & & & 4 & 0 & 2 & 2 & 0 & 0 \\
        \hline
    \end{tabular}
\end{table}


    \endgroup
    \docline

\end{document}
